\documentclass[twocolumn, 11pt]{article}

\usepackage[brazilian]{babel}
\usepackage[utf8]{inputenc}
\usepackage{indentfirst}

\usepackage{mathtools}
\usepackage{color}
\usepackage{geometry}
\usepackage{listings}

% Tamanho das margens:
\geometry{
	a4paper,
	total={170mm,257mm},
	left=20mm,
	top=20mm,
}

% Definir cores personalizadas
\definecolor{codegreen}{rgb}{0,0.6,0}
\definecolor{codegray}{rgb}{0.5,0.5,0.5}
\definecolor{codepurple}{rgb}{0.58,0,0.82}
\definecolor{backcolour}{rgb}{0.961, 0.961, 0.961}
\definecolor{keywordcolour}{rgb}{0.8,0,0.4}
\definecolor{labelcolour}{rgb}{0.31, 0.4, 0.741}

% Configurar linguagem Assembly personalizada
\lstdefinelanguage{Assembly}{
    morekeywords={li, branch, mult, add, addi, jump, show, halt},
    keywordstyle=\color{keywordcolour}\bfseries,
    morekeywords=[2]{main:, loop:, fim_loop:, loop, fim_loop}, 
    keywordstyle=[2]\color{labelcolour}\bfseries, 
    morecomment=[l]{;},
    commentstyle=\color{codegreen}\itshape,
    morestring=[b]",
    stringstyle=\color{codepurple},
    sensitive=true,
    alsoletter=:
}

\lstdefinestyle{mystyle}{
    backgroundcolor=\color{backcolour},   
    basicstyle=\ttfamily\footnotesize,
    breakatwhitespace=false,         
    showspaces=false,                
    tabsize=2,
    language=Assembly
}

\lstset{style=mystyle}

\author{Andrieli Luci Gonçalves}
\title{Relatório do Trabalho Prático de Projetos Digitais e Microprocessadores - CI1210}

\begin{document}

\maketitle        

\section{Introdução}

Este relatório descreve o processo de construção de um processador simplificado, baseado na Arquitetura MIPS. O objetivo foi implementar um processador capaz de buscar e executar instruções de lógico-aritméticas e de controle de fluxo (como salto) a cada ciclo. Para tal, foi utilizado o simulador \textit{Logisim Evolution 3.8.0}.

As instruções implementadas foram criadas para calcular a soma dos termos de uma progressão aritmética (PA), demonstrando o funcionamento do processador em relação à Arquitetura de Conjunto de Instruções (ISA).

\section{Microprocessador}

O projeto do processador é composto por dois blocos principais: o circuito de dados, que inclui a Unidade Lógica e Aritmética (ULA) e o banco de registradores; e o circuito de controle, que abrange o contador de programa (PC), a memória de instruções e a memória de controle -- além de circuitos auxiliares em ambos.

\subsection{Banco de registradores}

O componente foi construído com 16 registradores de 32 bits, formados pela união de oito registradores de quatro bits, cada um composto por quatro \textit{flip-flops} tipo D com \textit{Enable}.

A escrita no banco de registradores é controlada por um demultiplexador com 16 saídas. Ele possui uma entrada $WriteReg$ que habilita a escrita e um seletor $C_{0:3}$ para escolher o registrador onde a escrita acontecerá.

A leitura utiliza dois multiplexadores de 32 bits, cujos seletores de quatro bits, $RA$ e $RB$, indicam os registradores a serem lidos.

\subsection{Contador de programa (PC)}

O contador de programa (\textit{PC}) foi implementado de forma similar aos componentes do banco de registradores, exceto pela não utilização do \textit{Enable} nos flip-flops. Devido à limitação de 24 bits nos endereços da memória de instruções no \textit{Logisim Evolution}, o \textit{PC} possui em sua saída um distribuidor que coleta apenas os seus 24 bits menos significativos.

Sua função é controlar o fluxo das instruções, indicando a posição atual de execução na sequência programada.

\subsection{Memória de instrução}

A memória de instrução foi implementada utilizando o componente ROM do \textit{Logisim Evolution}, com larguras de 24 bits para endereços e 32 bits para dados. Na memória ROM, foram armazenadas as instruções do código em C que calcula a soma de uma progressão aritmética, convertidas previamente para Assembly, binário e hexadecimal, nesta ordem.

\subsubsection{Conversão para Assembly}

\begin{lstlisting}
main:
    li r1, 10   ; N - numero de termos
    li r2, 7    ; a - termo inicial
    li r3, 18   ; d - razao
    li r4, 0    ; soma - soma da PA 

    li r5, 0    ; contador
loop:
    branch r5, r1, fim_loop
    mult r6, r5, r3
    add r6, r6, r2
    add r4, r4, r6
    show r4

    addi r5, r5, 1	
    jump loop 	
fim_loop:
    show r4 
    halt
\end{lstlisting}

Para as conversões binárias, definiu-se um código para cada uma das instruções, iniciando em 0000 para o $nor$ e indo até 1110 em $halt$. Além disso, cada um dos 16 registradores também foi contemplado com um código, seguindo a contagem padrão em binário. O mesmo foi realizado com os valores imediatos. 

\subsection{Memória de controle}

Para a memória de controle, também foi utilizado o componente ROM da ferramenta; neste caso, a largura de endereço equivale a quatro bits e a largura de dados, nove bits. Desse modo, é possível armazenar os sinais de controle necessários para guiar a execução das instruções. Tais sinais foram definidos considerando a funcionalidade de cada instrução:

\begin{itemize}
  \item $ALUSrc$: ULA deve receber um operando imediato ou proveniente de um registrador; 
  \item $ALUOp$: operação a ser realizada pela ULA;
  \item $Branch$: salto condicional (com base em uma comparação entre dois valores);
  \item $Jump$: salto incondicional;
  \item $WriteReg$: controla se o resultado da operação será armazenado em um registrador;
  \item $ImmToReg$: controla se o dado a ser carregado em um registrador provém de um valor imediato ou da ULA;
  \item $EDisplay$: determina quando o resultado aparecerá em tela.
\end{itemize}

\subsection{Unidade Lógica e Aritmética (ULA)}

A Unidade Lógica e Aritmética é composta por sete instruções. A primeira (000) e a segunda (001), que representam respectivamente a soma e a subtração, são formadas por somadores de 32 bits (gerados a partir da combinação de somadores de 1 bit). Para multiplicação (010), $AND$ (011), $OR$ (100) e $XOR$ (101), utilizou-se diretamente os componentes do \textit{Logisim Evolution}. 

A operação de deslocamento à esquerda (110) foi implementada com 32 multiplexadores. A entrada $A_{0:31}$ representa o dado a ser deslocado e $B_{0:31}$, a quantidade de bits de deslocamento, sendo considerados apenas os cinco bits menos significativos de $B$ para a operação. No multiplexador correspondente à saída $S31$, todas as entradas recebem os bits de $A$. Para $S30$, a última entrada é substituída por 0. Esse padrão continua até $S0$, onde apenas a primeira entrada contém dados de $A$, com as demais em 0. Para deslocamentos superiores a 31 bits, foi criado um circuito combinacional apenas com a porta lógica $OR$ que verifica se qualquer bit posterior ao quinto de $B$ é 1.

Na região externa da ULA, um multiplexador de 32 bits seleciona a operação a ser realizada com base no seletor $Funct$, definindo a saída $S$. Além disso, há uma saída chamada $Zero$, que é gerada pela negação de todos os bits de $S$ combinada por 11 portas lógicas AND -- o propósito desta \textit{flag} é indicar se a saída é igual (ou não) a zero.

\subsection{Circuitos auxiliares}

Os extensores foram moldados a partir de distribuidores: A entrada $A_{0:15}$ é expandida em 16 partes de 1 bit, ao passo que a saída $B_{0:31}$ é expandida em 32 partes de 1 bit. No caso do extensor de zero, os bits adicionais são preenchidos com 0. Já no extensor de sinal, o bit mais significativo de $A$ é replicado para completar a saída $B$.

Para o funcionamento do processador, também foram utilizados dois somadores de 32 bits, ambos conectados ao PC. O primeiro somador tem como operandos $A_{0:31} = 1$ e $B_{0:31} = \text{endereço atual do PC}$; dessa forma, seu objetivo é incrementar em uma unidade o PC para que a próxima instrução aconteça. O segundo somador tem como operandos $A_{0:31} = \text{Const com sinal extendido}$ e $B_{0:31} = \text{endereço atual do PC}$; assim, sua funcionalidade é provocar saltos (condicionais ou incondicionais) ao decorrer da execução.

Também associados à atuação do PC e ao controle dos saltos, utilizou-se duas portas lógicas ($AND$ e $OR$): a primeira é ligada quando $Branch$ está ativo e o resultado contido na ULA equivale a zero; a segunda é ligada quando a primeira porta ou a \textit{flag} $Jump$ estão ativas.

Ademais, para exibir os resultados parciais e finais da soma da progressão aritmética, foi usado um componente de \textit{display} do próprio \textit{Logisim Evolution}. Para garantir que o resultado permanecesse visível durante toda a execução, houve a adição de um registrador para armazenar o estado anterior do \textit{display}.

Por fim, trabalhou-se também com três multiplexadores, todos associados à seleção de conteúdos específicos com base nos sinais de controle do microprocessador.

\end{document}